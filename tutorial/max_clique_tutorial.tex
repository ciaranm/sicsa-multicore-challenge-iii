% vim: set spell spelllang=en tw=100 et sw=4 sts=4 :

\documentclass[a4paper]{article}

\usepackage[inner=1in,outer=1in,top=1in,bottom=1.4in]{geometry}
\usepackage{microtype}                 % the pretty
\usepackage{titlesec}                  % muck around with titles
\usepackage{caption}
\usepackage{complexity}                % \P, \NP etc
\usepackage{varioref}                  % \vref
\usepackage{hyperref}                  % clicky links
\usepackage{cleveref}                  % no need to type Figure etc
\usepackage{tikz}                      % For pretty pictures
\usepackage{amsmath}                   % \text
\usepackage{wrapfig}                   % wrapping figures
\usepackage{pdflscape}
\usepackage[space]{cite}
\usepackage{enumitem}
\usepackage{afterpage}                 % \afterpage
\usepackage{placeins}                  % \FloatBarrier
\usepackage[ruled,vlined]{algorithm2e} % algorithms

\usetikzlibrary{arrows, shadows, calc, positioning, decorations, decorations.pathmorphing,
    decorations.pathreplacing, patterns, tikzmark, fit}

% cref style
\crefname{figure}{Figure}{Figures}
\Crefname{figure}{Figure}{Figures}
\crefname{table}{Table}{Tables}
\Crefname{table}{Table}{Tables}
\crefname{part}{Part}{Parts}
\Crefname{part}{Part}{Parts}
\crefname{section}{Section}{Sections}
\Crefname{section}{Section}{Sections}
\crefname{algocf}{algorithm}{algorithms}
\Crefname{algocf}{Algorithm}{Algorithms}

% captions
\captionsetup[figure]{labelfont=bf}

\setlength{\belowcaptionskip}{6pt}

% titles
\newlength\marginnumberslen
\setlength\marginnumberslen{\dimexpr\oddsidemargin+0.4in+\hoffset\relax}

\titleformat{name=\section}{\normalfont\Large\bfseries}{\llap{\hspace*{-\marginnumberslen}\thesection\hfill}}{0em}{}[{\vspace*{-\baselineskip}\medskip\hspace*{-\marginnumberslen}\textcolor{safelightgrey}{\titlerule}}]

\titleformat{name=\subsection}{\normalfont\large\bfseries}{\llap{\hspace*{-\marginnumberslen}\thesubsection\hfill}}{0em}{}[{\vspace*{-\baselineskip}\medskip\hspace*{-\marginnumberslen}\textcolor{safenearlywhite}{\titlerule}}]


% bibliography
\makeatletter
\let\old@biblabel\@biblabel
\def\@biblabel#1{\llap{\hspace*{-\marginnumberslen}\old@biblabel{#1}\hfill}}%

\let\orig@thebibliography\thebibliography
\renewcommand{\thebibliography}{\labelsep=\z@ \leftmargin=\z@ \orig@thebibliography}
\makeatother

% colours
\definecolor{safelightblue}{rgb}{0.65098, 0.807843, 0.890196}
\definecolor{safedarkblue}{rgb}{0.121569, 0.470588, 0.705882}
\definecolor{safeverylightblue}{rgb}{0.857843, 0.931961, 0.996078}
\definecolor{safeverydarkblue}{rgb}{0.007843, 0.219608, 0.345098}
\definecolor{safelightorange}{rgb}{0.996078, 0.931961, 0.857843}
\definecolor{safelightgreen}{rgb}{0.698039, 0.87451, 0.541176}
\definecolor{safemediumorange}{rgb}{0.74902, 0.505882, 0.490196}
\definecolor{safelightgreen}{rgb}{0.698039, 0.87451, 0.541176}
\definecolor{safedarkgreen}{rgb}{0.2, 0.627451, 0.172549}
\definecolor{safepurple}{rgb}{0.458824, 0.439216, 0.701961}
\definecolor{safedarkorange}{rgb}{0.34902, 0.176471, 0.015686}
\definecolor{safelightgrey}{rgb}{0.7, 0.7, 0.7}
\definecolor{safenearlywhite}{rgb}{0.9, 0.9, 0.9}
\definecolor{safereallynearlywhite}{rgb}{0.93, 0.93, 0.93}

% showframe haxx
\makeatletter
\def\Gm@hrule{\color{safereallynearlywhite}\hrule height 0.2pt depth\z@ width\textwidth}
\def\Gm@hruled{}
\newcommand*{\gmshow@textheight}{\textheight}
\newdimen\gmshow@@textheight
\g@addto@macro\landscape{\gmshow@@textheight=\hsize\renewcommand*{\gmshow@textheight}{\gmshow@@textheight}}
\def\Gm@vrule{\color{safereallynearlywhite}\vrule width 0.2pt height\gmshow@textheight depth\z@}
\makeatother

% drawing styles
\tikzset{vertex/.style={draw, circle, inner sep=0pt, minimum size=0.5cm, font=\small\bfseries}}
\tikzset{notvertex/.style={vertex, color=white, text=black}}
\tikzset{plainvertex/.style={vertex}}
\tikzset{selectedvertex/.style={vertex, fill=safedarkblue}}
\tikzset{delvertex/.style={vertex, dotted, color=safelightgrey}}
\tikzset{vertexc1/.style={vertex, fill=safelightblue}}
\tikzset{vertexc2/.style={vertex, fill=safedarkblue, text=safenearlywhite}}
\tikzset{vertexc3/.style={vertex, fill=safelightorange}}
\tikzset{vertexc4/.style={vertex, fill=safedarkorange, text=safenearlywhite}}
\tikzset{vertexc5/.style={vertex, fill=safedarkgreen}}

\tikzset{edge/.style={color=safelightgrey}}
\tikzset{bedge/.style={ultra thick}}
\tikzset{deledge/.style={dotted, color=safelightgrey}}
\tikzset{edgel1/.style={color=safeverydarkblue}}
\tikzset{edgel2/.style={color=safemediumorange}}
\tikzset{edgel3/.style={ultra thick, color=safedarkblue}}
\tikzset{edgel4/.style={ultra thick, color=safelightorange}}

\tikzset{processarrow/.style={->, very thick, decorate, decoration={snake, post length=0.5mm}}}
\tikzset{brace/.style={decorate, decoration={brace}, thick}}
\tikzset{bracem/.style={decorate, decoration={brace, mirror}, thick}}
\tikzset{label/.style={font=\small}}

% Use these colours to replace the obnoxious defaults used for links in PDFs
\hypersetup{
    colorlinks,
    citecolor=safedarkgreen,
    linkcolor=safedarkblue,
    urlcolor=safepurple}

% \input isn't expandable. yay.
\makeatletter\newcommand*\inputhaxx[1]{\@@input #1}\makeatother

% notation
\newcommand{\vertexset}{V}
\newcommand{\edgeset}{E}
\newcommand{\neighbourhood}{N}
\newcommand{\modularproduct}{\operatorname{\triangledown}}
\newcommand{\graphcomplement}[1]{\shortoverline{#1}}
\newcommand{\Aut}[1]{\operatorname{Aut}{#1}}
\newcommand{\End}[1]{\operatorname{End}{#1}}
\newcommand{\compose}{\circ}
\newcommand{\inverse}{^{-1}}

\newcommand{\Cbest}{C^\star}
\newcommand{\Lbest}{L^\star}
\newcommand{\first}{\mathit{first}}
\newcommand{\bounds}{\mathit{bounds}}
\newcommand{\budget}{\mathit{budget}}
\newcommand{\order}{\mathit{order}}
\newcommand{\colour}{\mathit{colour}}
\newcommand{\clique}{\mathit{clique}}
\newcommand{\uncoloured}{\mathit{uncoloured}}
\newcommand{\remaining}{\mathit{remaining}}
\newcommand{\colourable}{\mathit{colourable}}
\newcommand{\cliqueable}{\mathit{cliqueable}}
\newcommand{\singletons}{\mathit{singletons}}
\newcommand{\expand}{\FuncSty{expand}}
\newcommand{\colourOrder}{\FuncSty{colourOrder}}
\newcommand{\colourOrderDefer}{\FuncSty{colourOrderDefer}}
\newcommand{\cliquePartitionDeferOrder}{\FuncSty{cliquePartitionDeferOrder}}
\newcommand{\vrej}{v_{\mathit{rej}}}
\newcommand{\kwunset}{\KwSty{unset}}

\newcommand{\mclabel}[1]{\label{line:mc:#1}}
\newcommand{\mcline}[1]{line~\ref{line:mc:#1}}
\newcommand{\mclinerange}[2]{lines~\ref{line:mc:#1} to~\ref{line:mc:#2}}
\newcommand{\mcmark}[1]{\tikzmark{mark:mc:#1}}

\makeatletter
\def\blfootnote{\xdef\@thefnmark{}\@footnotetext}
\makeatother

\title{Algorithms for the Maximum Clique Problem, \\ and How to Parallelise Them}

\author{
    Ciaran McCreesh%
    \\ \href{mailto:c.mccreesh.1@research.gla.ac.uk}{\nolinkurl{c.mccreesh.1@research.gla.ac.uk}}
}

\begin{document}

\makeatletter
\vspace*{1cm}\noindent{\centering\Large\bfseries \@title \par}
\vspace*{1cm}\noindent{\centering \@author \par}
\vspace*{0.5cm}\noindent{\centering \@date \par}
\vspace*{0.5cm}
\makeatother

\blfootnote{Parts of this text and the associated figures are based upon material which has
previously been published or submitted for publication
\cite{McCreesh:2014.labelled,McCreesh:2014.reorder,McCreesh:2014.parallel}.}

\section{Introduction}

\begingroup\setlength\intextsep{0pt}% hack for fixing space above the figure.
\begin{wrapfigure}{R}{0.4\textwidth}
    \centering
    \begin{tikzpicture}
        \newcount \myc
        \foreach \n in {1, ..., 8}{
            \myc=\n \advance\myc by -1 \multiply\myc by -360 \divide\myc by 8 \advance\myc by 90
            \advance\myc by 22.5
            \ifthenelse{\n = 1 \OR \n = 4 \OR \n = 6 \OR \n = 7}{
                \node[selectedvertex] (N\n) at (\the\myc:1.25) {\n};
            }{
                \node[vertex] (N\n) at (\the\myc:1.25) {\n};
            }
        }

        \draw [edge] (N1) -- (N2);
        \draw [edge] (N1) -- (N3);
        \draw [edge] (N1) -- (N8);
        \draw [edge] (N2) -- (N3);
        \draw [edge] (N2) -- (N7);
        \draw [edge] (N3) -- (N5);
        \draw [edge] (N3) -- (N6);
        \draw [edge] (N4) -- (N5);
        \draw [edge] (N5) -- (N8);

        \draw [bedge] (N1) -- (N4);
        \draw [bedge] (N1) -- (N6);
        \draw [bedge] (N1) -- (N7);
        \draw [bedge] (N4) -- (N6);
        \draw [bedge] (N4) -- (N7);
        \draw [bedge] (N6) -- (N7);
    \end{tikzpicture}
    \caption{A graph, with its maximum clique $\{1, 4, 6, 7\}$ shown.}
    \label{figure:maxclique}
\end{wrapfigure}

A \emph{clique} in a graph is a subset of vertices, where every vertex in the set is adjacent to
every other in the set (this can be thought of as a collection of people, all of whom know each
other). We illustrate this in \cref{figure:maxclique}. Finding the largest clique in a given graph
is the \emph{maximum clique problem}. Applications include biochemistry
\cite{Butenko:2006,Eblen:2011,Eblen:2012,Depolli:2013}, control of autonomous vehicles
\cite{Regula:2013}, coding and communications theory \cite{Bomze:1999}, document search
\cite{Okubo:2006} and computer vision \cite{Pelillo:1998}.

An \emph{independent set} is a subset of vertices, where no two vertices in the set are
adjacent---in other words, it is a clique in the complement graph. We will discuss maximum clique
here rather than maximum independent set, because there is a large set of standard benchmark
instances that are formulated as clique problems. (The minimum vertex cover problem is also
equivalent.)

The maximum clique and maximum independent set problems are \NP-hard \cite{Garey:1979}. However,
there are practical exact algorithms which work on a wide range of inputs in a reasonable amount of
time. We describe a slightly simplified version of a widely used algorithm for the maximum clique
problem, and discuss how it may be parallelised. This is \emph{not} a full discussion of the
state-of-the-art in maximum clique solving or a full literature review---we believe that what we
present here it is close enough to be an interesting and fair problem, but not too complicated to
make implementations unrealistically challenging.

\endgroup % hack for fixing space above the figure

\section{A Branch and Bound Algorithm}

In \vref{algorithm:maxClique} we present a basic branch and bound algorithm for the maximum clique
problem. This is a simplification of a series of algorithms due to Tomita et al.\
\cite{Tomita:2003,Tomita:2007,Tomita:2010}. We refer to a computational study by Prosser
\cite{Prosser:2012} for a comparison of these algorithms; what we describe initially is Prosser's
``MCSa1'' variation.

\paragraph{Branching} Let $v$ be some vertex in our graph. Any clique either contains only
$v$ and possibly some vertices adjacent to $v$, or does not contain $v$. Thus we may build up
potential solutions by recursively selecting a vertex, and branching on whether or not to include
it. We store our growing clique in a variable $C$, and vertices which may potentially be added to
$C$ are stored in a variable $P$.  Initially $C$ is empty, and $P$ contains every vertex
(\mcline{initial}).

The $\expand$ function is our main recursive procedure. Inside a loop
(\mclinerange{loopstart}{loopend}), we select a vertex $v$ from $P$ (\mcline{v}). First we consider
including $v$ in $C$ (\mclinerange{vinstart}{vinend}). We produce a new $P'$ from $P$ by rejecting
any vertices which are not adjacent to $v$ (\mcline{pprime})---this is sufficient to ensure that
$P'$ contains only vertices adjacent to \emph{every} vertex in $C$. If $P'$ is not empty, we may
potentially grow $C$ further, and so we recurse (\mcline{recurse}).  Having considered $v$ being in
the clique, we then reject $v$ (\mclinerange{vnotinstart}{vnotinend}) and repeat.

\paragraph{Bounding} Now we discuss a bound.  If we can colour a graph using $k$ colours, giving
adjacent vertices different colours, then we know that the graph cannot contain a clique of size
greater than $k$ (each vertex in a clique must be given a different colour---see
\cref{figure:colour}). This gives us a bound on how much further $C$ could grow, using only the
vertices remaining in $P$. To make use of this bound, we keep track of the largest feasible solution
we have found so far (called the \emph{incumbent}), which we store in $\Cbest$. Initially $\Cbest$
is empty (\mcline{cbest}).  Whenever we find a new feasible solution, we compare it with $\Cbest$,
and if it is larger, we unseat the incumbent (\mcline{unseat}).

\begin{wrapfigure}{R}{0.4\textwidth}
    \centering
    \begin{tikzpicture}
        \newcount \myc
        \foreach \n in {1, ..., 8}{
            \myc=\n \advance\myc by -1 \multiply\myc by -360 \divide\myc by 8 \advance\myc by 90
            \advance\myc by 22.5
            \ifthenelse{\n = 1 \OR \n = 5}{
                \node[vertexc1] (N\n) at (\the\myc:1.25) {\n};
            }{}

            \ifthenelse{\n = 2 \OR \n = 6 \OR \n = 8}{
                \node[vertexc2] (N\n) at (\the\myc:1.25) {\n};
            }

            \ifthenelse{\n = 3 \OR \n = 4}{
                \node[vertexc3] (N\n) at (\the\myc:1.25) {\n};
            }

            \ifthenelse{\n = 7}{
                \node[vertexc4] (N\n) at (\the\myc:1.25) {\n};
            }
        }

        \draw [edge] (N1) -- (N2);
        \draw [edge] (N1) -- (N3);
        \draw [edge] (N1) -- (N8);
        \draw [edge] (N2) -- (N3);
        \draw [edge] (N2) -- (N7);
        \draw [edge] (N3) -- (N5);
        \draw [edge] (N3) -- (N6);
        \draw [edge] (N4) -- (N5);
        \draw [edge] (N5) -- (N8);

        \draw [bedge] (N1) -- (N4);
        \draw [bedge] (N1) -- (N6);
        \draw [bedge] (N1) -- (N7);
        \draw [bedge] (N4) -- (N6);
        \draw [bedge] (N4) -- (N7);
        \draw [bedge] (N6) -- (N7);
    \end{tikzpicture}
    \caption{The relationship between cliques and colouring: each vertex in any clique must have a
    different colour.}
    \label{figure:colour}
\end{wrapfigure}

For each recursive call, we produce a constructive colouring of the vertices in $P$
(\mcline{colour}), using the $\colourOrder$ function. This process produces an array $\order$ which
contains a permutation of the vertices in $P$, and an array of bounds, $\bounds$, in such a way that
the subgraph induced by the first $i$ vertices of $\order$ may be coloured using $\bounds[i]$
colours. The $\bounds$ array is non-decreasing ($\bounds[i + 1] \ge \bounds[i]$), so if we iterate
over $\order$ from right to left, we can avoid having to produce a new colouring for each choice of
$v$. We make use of the bound on \mcline{bound}: if the size of the growing clique plus the number
of colours used to colour the vertices remaining in $P$ is not enough to unseat the incumbent, we
abandon search and backtrack.

The $\colourOrder$ function performs a simple greedy colouring. We select a vertex (\mcline{cv}) and
give it the current colour (\mcline{cgive}). This step is repeated until no more vertices may be
given the current colour without causing a conflict (\mclinerange{cloopstart}{cloopend}). We then
proceed with a new colour (\mcline{cnewcolour}) until every vertex has been coloured
(\mclinerange{cloopoutstart}{cloopoutend}). Vertices are placed into the $\order$ array in the order
in which they were coloured, and the $i$th entry of the $\bounds$ array contains the number of
colours used at the time the $i$th vertex in $\order$ was coloured. This process is illustrated in
\vref{figure:colourOrder}.

\paragraph{Initial Vertex Orderings} The order in which vertices are coloured can have a substantial
effect upon the colouring produced.  Here we will select vertices in a static non-increasing degree
order, tie-breaking on vertex number (for reproducibility). This is done by permuting the graph at
the top of search (\mcline{permute}), so vertices are simply coloured in numerical order. This
assists with the bitset encoding, which we discuss below.

\section{Bit Parallelism}

For the maximum clique problem, San Segundo et al.\ \cite{SanSegundo:2011,SanSegundo:2013} observed
that using a bitset encoding for SIMD-like local parallelism could speed up an implementation by a
factor of between two to twenty, without changing the steps taken. This is done as follows: $P$
should be a bitset, and the graph should be represented using an adjacency bitset for each vertex
(this representation may be created when $G$ is permuted, on \mcline{permute}). Then \mcline{pprime}
is simply a bitwise-and operation, behaving like intersection.

Most importantly, the $\uncoloured$ and $\colourable$ variables in $\colourOrder$ are also bitsets,
and the filtering on \mcline{removeadjacent} is simply a bitwise and-with-complement operation.
Find the first vertex in $\colourable$ on line \mcline{cv} is provided as a dedicated instruction in
modern hardware.

Note that $C$ should not be stored as a bitset. Instead, it should be an array. Adding a vertex to
$C$ on \mcline{vtoc} may be done by appending to the array, and when removing a vertex from $C$ on
\mcline{vfromc} we simply remove the last element---this works because $C$ is used like a stack.

\begin{algorithm}[p]\DontPrintSemicolon
    \begin{tikzpicture}[overlay, remember picture]
        \coordinate (accept1) at ({$(pic cs:mark:mc:bracesright)+(1, 0)$}|-{$(pic cs:mark:mc:accept1)+(0,-0.1)$});
        \coordinate (accept2) at ({$(pic cs:mark:mc:bracesright)+(1, 0)$}|-{$(pic cs:mark:mc:accept2)$});
        \coordinate (acceptt) at ($(accept1)!0.5!(accept2)$);
        \draw [brace] (accept1) -- (accept2);
        \node [right = 0.2 of acceptt] { accept $v$ };

        \coordinate (reject1) at ({$(pic cs:mark:mc:bracesright)+(1, 0)$}|-{$(pic cs:mark:mc:reject1)+(0,-0.1)$});
        \coordinate (reject2) at ({$(pic cs:mark:mc:bracesright)+(1, 0)$}|-{$(pic cs:mark:mc:reject2)$});
        \coordinate (rejectt) at ($(reject1)!0.5!(reject2)$);
        \draw [brace] (reject1) -- (reject2);
        \node [right = 0.2 of rejectt] { reject $v$ };
    \end{tikzpicture}

    \nl $\FuncSty{maximumClique}$ :: (Graph $G$) $\rightarrow$ Vertex Set \;
    \nl \Begin{
        \nl permute $G$ so that vertices are in non-increasing degree order, tie-breaking on vertex number\mclabel{permute} \;
        \nl $\KwSty{global}$ $\Cbest$ $\gets$ $\emptyset$ \mclabel{cbest} \;
        \nl $\expand$($\emptyset$, $\vertexset(G)$) \mclabel{initial} \;
        \nl $\KwSty{return}$ $\Cbest$ (unpermuted) \;
    }
    \;

    \nl $\expand$ :: (Vertex Set $C$, Vertex Set $P$) \;
    \nl \Begin{
        \nl ($\order$, $\bounds$) $\gets$ $\colourOrder$($P$) \mclabel{colour} \;
        \nl \For{$i$ $\gets$ $|P|$ $\KwSty{downto}$ 1\mclabel{loopstart}}{
            \nl \lIf{\textnormal{$|C|$ + $\bounds[i]$ $\le$ $|\Cbest|$}\mclabel{bound}}{$\KwSty{return}$}
            \nl $v$ $\gets$ $\order[i]$ \mclabel{v} \mcmark{accept1} \;
            \nl add $v$ to $C$ \mclabel{vinstart} \mclabel{vtoc} \;
            \nl \lIf{\textnormal{$|C| > |\Cbest|$}}{$\Cbest$ $\gets$ $C$\mclabel{unseat}}
            \nl $P'$ $\gets$ the vertices in $P$ that are adjacent to $v$ \mclabel{pprime} \mcmark{bracesright} \;
            \nl \lIf{$P'$ $\ne$ $\emptyset$}{$\expand$($C$, $P'$) \mclabel{recurse} \mcmark{accept2} \mcmark{reject1}} \mclabel{vinend}
            \nl remove $v$ from $C$ \mclabel{vnotinstart} \mclabel{vfromc} \;
            \nl remove $v$ from $P$ \mclabel{vnotinend} \mclabel{loopend} \mcmark{reject2} \;
        }
    }
    \;

    \nl $\colourOrder$ :: (Vertex Set $P$) $\rightarrow$ (Vertex Array, Int Array) \;
    \nl \Begin{
        \nl ($\order$, $\bounds$) $\gets$ ($[]$, $[]$) \;
        \nl $\uncoloured$ $\gets$ $P$ \;
        \nl $\colour$ $\gets$ $1$ \;
        \nl \While{$\uncoloured$ $\ne$ $\emptyset$\mclabel{cloopoutstart}}{
            \nl $\colourable$ $\gets$ $\uncoloured$ \;
            \nl \While{$\colourable$ $\ne$ $\emptyset$\mclabel{cloopstart}}{
                \nl $v$ $\gets$ the first vertex of $\colourable$ \mclabel{cv} \;
                \nl append $v$ to $\order$, and $\colour$ to $\bounds$ \mclabel{cgive} \;
                \nl remove $v$ from $\uncoloured$ and from $\colourable$ \;
                \nl remove from $\colourable$ all vertices adjacent to $v$
                \mclabel{removeadjacent}\mclabel{cloopend} \;
            }
            \nl add $1$ to $\colour$ \mclabel{cnewcolour}\mclabel{cloopoutend}
        }
        \nl $\KwSty{return}$ ($\order$, $\bounds$) \;
    }

    \caption{A basic maximum clique algorithm.}
    \label{algorithm:maxClique}
\end{algorithm}

\begin{figure}[p]
    \centering
    \begin{tikzpicture}
        \newcount \myc
        \foreach \n in {1, ..., 8}{
            \myc=\n \advance\myc by -1 \multiply\myc by -360 \divide\myc by 8 \advance\myc by 90 \advance\myc by 22.5
            \ifthenelse{\n = 1 \OR \n = 3}{
                \node[vertexc1] (N\n) at (\the\myc:1.2) {\n};
            }{}
            \ifthenelse{\n = 2 \OR \n = 4 \OR \n = 8} {
                \node[vertexc2] (N\n) at (\the\myc:1.2) {\n};
            }{}
            \ifthenelse{\n = 5 \OR \n = 7}{
                \node[vertexc3] (N\n) at (\the\myc:1.2) {\n};
            }{}
            \ifthenelse{\n = 6}{
                \node[vertexc4] (N\n) at (\the\myc:1.2) {\n};
            }{}
        }

        \draw [edge] (N1) -- (N2);
        \draw [edge] (N1) -- (N4);
        \draw [edge] (N1) -- (N5);
        \draw [edge] (N1) -- (N6);
        \draw [edge] (N1) -- (N8);
        \draw [edge] (N2) -- (N3);
        \draw [edge] (N2) -- (N7);
        \draw [edge] (N3) -- (N4);
        \draw [edge] (N3) -- (N5);
        \draw [edge] (N3) -- (N7);
        \draw [edge] (N4) -- (N5);
        \draw [edge] (N4) -- (N6);
        \draw [edge] (N5) -- (N6);
        \draw [edge] (N7) -- (N8);

        \draw [processarrow] (1.6, 0) -> (2.7, 0);

        \coordinate (Ms) at (4.3, 0.35);
        \node[right = 0.0 of Ms, vertexc1] (M1) {1};
        \node[right = 0.2 of M1, vertexc1] (M2) {3};
        \node[right = 0.2 of M2, vertexc2] (M3) {2};
        \node[right = 0.2 of M3, vertexc2] (M4) {4};
        \node[right = 0.2 of M4, vertexc2] (M5) {8};
        \node[right = 0.2 of M5, vertexc3] (M6) {5};
        \node[right = 0.2 of M6, vertexc3] (M7) {7};
        \node[right = 0.2 of M7, vertexc4] (M8) {6};

        \node[left = 0.2 of M1, label] {$\order$:};
        \draw[brace] ($(M1.north west)+(0.0,0.3)$) -- ($(M8.north east)+(0.0,0.3)$);
        \node[anchor=south, label] at ($(M1)!0.5!(M8)$)[yshift=0.7cm] { Vertices in colour order };

        \coordinate (Bs) at (4.3, -0.35);
        \node[right = 0.0 of Bs, notvertex] (B1) {1};
        \node[right = 0.2 of B1, notvertex] (B2) {1};
        \node[right = 0.2 of B2, notvertex] (B3) {2};
        \node[right = 0.2 of B3, notvertex] (B4) {2};
        \node[right = 0.2 of B4, notvertex] (B5) {2};
        \node[right = 0.2 of B5, notvertex] (B6) {3};
        \node[right = 0.2 of B6, notvertex] (B7) {3};
        \node[right = 0.2 of B7, notvertex] (B8) {4};

        \node[left = 0.2 of B1, label] {$\bounds$:};
        \draw[brace] ($(B8.south east)+(0.0,-0.2)$) -- ($(B1.south west)+(0.0,-0.2)$);
        \node[anchor=north, label] at ($(B1)!0.5!(B8)$)[yshift=-0.6cm] { Number of colours used };
    \end{tikzpicture}

    \caption[How the $\colourOrder$ function works]{How the $\colourOrder$ function works. The graph
        on the left has been coloured greedily: vertices 1 and 3 were given the first colour, then
        vertices 2, 4 then 8 were given the second colour, then vertices 5 and 7 were given the
        third colour, then vertex 6 was given the fourth colour. On the right, we show the $\order$
        array, which contains the vertices in the order which they were coloured. Below, the $\bounds$
        array, containing the number of colours used so far.}
    \label{figure:colourOrder}
\end{figure}

\section{Thread Parallelism}

To make use of multi-core processors, one option is to just use threads to speed up the bit
parallelism. However, this would require considerable synchronisation, and we have not seen this
done successfully. Another option, which has been implemented independently at least twice
\cite{McCreesh:2013,Depolli:2013,McCreesh:2014.parallel}, is to use threads to do the search in
parallel. Here is how: we can view the recursive calls to $\expand$ as forming a tree. We can then
explore different parts of this tree in parallel, ignoring the left-to-right dependency. There are
at least three issues which must be addressed for this approach.

\paragraph{Sharing the incumbent} The $\Cbest$ variable must be shared between every thread.  One
thread could find a good solution quickly, which would allow a lot of the search space to be
eliminated due to the bound, but if other threads do not know this, they will be doing wasted work.
In the worst case, this can lead to an overall slowdown.

\paragraph{Work balance versus overhead} Published implementations of these algorithms can make
around $10^6$ recursive calls to $\expand$ per second (and this time is dominated by the colouring
step); standard benchmark problems can take between $10^6$ and $10^{10}$ recursive calls to solve.
Thus it may not be feasible, from an overhead perspective, to treat every single recursive call as a
potential parallel task.

On the other hand, subproblems can be of vastly different sizes, so splitting into a few large tasks
might not give a good enough balance. One approach, which works reasonably well for most (but not
all) benchmark instances for modest numbers of cores, is to treat a subtree as a potential
subproblem only if $|C| = 1$. If these subproblems are allocated dynamically, the irregularity
usually (but not always) sorts itself out \cite{Regin:2013,Depolli:2013,McCreesh:2013}.

\paragraph{Search order} It is important to prioritise subproblems, so that subtrees are explored in
roughly the same order as they would be sequentially---otherwise, it is possible that a parallel
implementation could spend all its time exploring parts of the search tree that would have been
eliminated in the sequential run \cite{Trienekens:1990,deBruin:1995}.  There is also an interaction
between work balance and diversity in search: improving the work balance but changing the search
order can give substantially worse speedups in some circumstances.  We have discussed this issue in
more depth elsewhere \cite{McCreesh:2014.parallel}.

\section{Testing and Benchmarking}

There are two standard benchmark suites used for testing maximum clique algorithms.  The first comes
from the Second DIMACS Implementation
Challenge\footnote{\url{http://dimacs.rutgers.edu/Challenges/}}. These graphs vary wildly in their
complexity. As a rough guide:

\paragraph{The ``brock'' family} These are due to Brockington and Culberson
\cite{Brockington:1996}. They are an attempt at camouflaging a known clique in a quasi-random graph
for cryptographic purposes, in a way that was resistant to certain early heuristic attacks.  There
are three subfamilies, ``brock200'', ``brock400'' and ``brock800''; the number denotes the number of
vertices in the graphs. Roughly speaking, for a good sequential exact maximum clique algorithm on modern
hardware, members of the ``brock200'' family can be solved in a second, the ``brock400'' family take
a few minutes, and the ``brock800'' family take around an hour.  In each case, there is a unique
maximum clique in these graphs, and the clique is larger than one would expect from a random graph
with the same density---this has interesting implications for parallel search.

\paragraph{The ``C'' family} These are random graphs. The ``.5'' family have all been solved,
although ``C2000.5'' takes a day, and ``C4000.5'' takes a month. The ``.9'' graphs from 500 upwards
are believed to be open problems; ``C125.9'' should take under a second, and ``C250.9'' should take
under an hour.

\paragraph{The ``DJSC'' family} ``DJSC500\_5'' should take a couple of seconds, and ``DSJC1000\_5''
should take a few minutes.

\paragraph{The ``c-fat'' family} These graphs are related to fault diagnosis for distributed
systems \cite{Berman:1990}. All are computationally trivial.

\paragraph{The ``gen'' and ``san'' families} The ``gen200'', ``san200'', ``sanr200'', ``san400'' and
``san1000'' graphs should take a few seconds; some of the ``sanr400'' graphs take a bit longer. The
``gen400'' graphs are extremely interesting from a parallelism perspective: once a solution has been
found, the proof of optimality is short, but finding an optimal solution is difficult.

\paragraph{The ``hamming'', ``johnson'' and ``keller'' families} These are all either too easy or
too hard to be interesting (although all are solvable using special knowledge of the structure of
the graphs \cite{Debroni:2011}).

\paragraph{The ``MANN'' family} These are clique formulations of the Steiner triple problem, due to
Mannino and Sassano \cite{Mannino:1995}. All have been solved: ``MANN\_a9'' and ``MANN\_a27'' are
not challenging, and ``MANN\_a45'' takes minutes; ``MANN\_a81'' was first solved as a general
maximum clique problem by the author and Prosser \cite{McCreesh:2013} in 31 days using 24 hardware
threads (although the solution was already known by other means). These graphs are very dense, which
sometimes gives atypical behaviour from algorithms. In particular, getting good work balance for
these graphs can be challenging.

\paragraph{The ``p\_hat'' family} These are random graphs with an unusually large degree spread,
created using the $\hat{p}$ generator by Soriano and Gendreau \cite{Soriano:1996}. The largest and
densest of these, ``p\_hat1500-3'', was first solved by the author and Prosser \cite{McCreesh:2013}
in 128 days using 32 hardware threads; ``p\_hat1000-3'' also takes multiple days to solve, but the
remainder of the graphs should take between a second and an hour.

\bigskip

\noindent Other interesting datasets include:

\paragraph{Benchmarks with Hidden Optimum Solutions for Graph Problems}
BHOSLIB\footnote{\url{http://www.nlsde.buaa.edu.cn/~kexu/benchmarks/graph-benchmarks.htm}} is a
collection of problem instances which are designed to be very hard to solve. Members of the
``frb30'' family should be solvable in under an hour, and the ``frb35'' family within a couple of
days.

\paragraph{Erd\H{o}s collaboration graphs} The Pajek
dataset\footnote{\url{http://vlado.fmf.uni-lj.si/pub/networks/data/}} includes larger, sparser
graphs. The Erd\H{o}s collaboration graphs are not computationally challenging, but do show how well
these algorithms can do on larger datasets.

\subsection{File Formats}

Except for the Erdos collaboration graphs, all of the standard benchmarks are in the ``DIMACS 2''
format. A simple graph looks like this:

\begin{verbatim}
    c This is a comment.
    c This graph looks a bit like a house.
    p edge 5 6
    e 1 2
    e 1 4
    e 1 5
    e 2 3
    e 3 4
    e 4 5
\end{verbatim}

Any ``c'' lines or blank lines are to be ignored.  There is a single ``p edge $v$ $e$'' line, where
$v$ is the number of vertices, and $e$ is the total number of edges in the graph (the $e$ figure is
sometimes omitted, wrong, or set to 0, so it is recommended to ignore this). Vertices are named $1$
to $v$.

For each edge, there is a line ``e $v_1$ $v_2$'', specifying that $v_1$ and $v_2$ are adjacent. Some
files specify edges in both directions, and some do not---either way, the graph is not directed.  It
can be assume that the first non-comment line is a ``p'' line, so the number of vertices is known
before any edges are specified.

\subsection{Verifying Sequential Code and Solutions}

One way to verify a sequential implementation is to count the number of calls made to
$\colourOrder$, and compare it to another implementation, such as the
code\footnote{\url{http://www.dcs.gla.ac.uk/~pat/maxClique/distribution/}} from Prosser's
computational study \cite{Prosser:2012}. (This assumes that for the initial ordering, tie-breaking
is done on vertex number; if this is done, our algorithm is Prosser's ``MCSa1''.) We have found that
doing this is extremely useful in catching the kinds of bugs where the implementation still gives
the right answer most of the time, but for the wrong reasons.

\FloatBarrier
\bibliographystyle{amsplain}
\bibliography{max_clique_tutorial}

\end{document}

